\section{SG\_\-Star Class Reference}
\label{class_s_g___star}\index{SG_Star@{SG\_\-Star}}
The primary star of the planetary system.  


{\tt \#include $<$SG\_\-Star.h$>$}

\subsection*{Public Member Functions}
\begin{CompactItemize}
\item 
{\bf SG\_\-Star} ()
\begin{CompactList}\small\item\em Constructor. \item\end{CompactList}\item 
{\bf $\sim$SG\_\-Star} ()\label{class_s_g___star_a1}

\begin{CompactList}\small\item\em Destructeur. \item\end{CompactList}\item 
void {\bf set\-Random\-Star} ()\label{class_s_g___star_a2}

\begin{CompactList}\small\item\em This function set the star parameters randomly. \item\end{CompactList}\item 
void {\bf set\-Mass} (long double mass)
\begin{CompactList}\small\item\em This function set the star mass, and calculate its luminosity, ecosphere, and life. \item\end{CompactList}\item 
void {\bf set\-Luminosity} (long double luminosity)
\begin{CompactList}\small\item\em This function set the star luminosity, and recalculate the ecosphere radius, and life. \item\end{CompactList}\item 
void {\bf set\-Ecosphere} (long double ecosphere)
\begin{CompactList}\small\item\em This function set the star ecosphere radius. \item\end{CompactList}\item 
void {\bf set\-Life} (long double life)
\begin{CompactList}\small\item\em This function set the star life time. \item\end{CompactList}\item 
void {\bf set\-Age} (long double age)
\begin{CompactList}\small\item\em This function set the age of the star. \item\end{CompactList}\item 
void {\bf set\-Magnitude} (long double magnitude)
\begin{CompactList}\small\item\em Calculate the luminosity of the star, based on its bolometric magnitude. \item\end{CompactList}\item 
long double {\bf get\-Body\-Temperature} (long double orbit\_\-radius, long double albedo)
\begin{CompactList}\small\item\em This function estimates the surface temperature for a body (ie: a planet), at a certain distance from the star. \item\end{CompactList}\item 
long double {\bf get\-Effective\-Temperature} (long double orbit\_\-radius, long double albedo)
\begin{CompactList}\small\item\em This function return the 'Effective Temperature' of a body. \item\end{CompactList}\item 
long double {\bf get\-Nearest\-Planet\-Orbit} ()
\begin{CompactList}\small\item\em Renvoie l'orbite minimale o\`{u} peut se former une planete. \item\end{CompactList}\item 
long double {\bf get\-Bode\-Planet\-Orbit} (int index)
\begin{CompactList}\small\item\em This function returns the orbit of the Nth planet of a planetary system. \item\end{CompactList}\item 
long double {\bf get\-Farthest\-Planet\-Orbit} ()
\begin{CompactList}\small\item\em Renvoie l'orbite maximale o\`{u} peut se former une planete. \item\end{CompactList}\item 
long double {\bf get\-Stellar\-Dust\-Limit} ()
\begin{CompactList}\small\item\em This function returns the radius of the stellar dust cloud surrounding the star. \item\end{CompactList}\item 
int {\bf get\-Planet\-Number} ()\label{class_s_g___star_a15}

\begin{CompactList}\small\item\em This function returns the number of planets orbiting around the star. \item\end{CompactList}\end{CompactItemize}
\subsection*{Public Attributes}
\begin{CompactItemize}
\item 
long double {\bf m\-Lum}\label{class_s_g___star_o0}

\begin{CompactList}\small\item\em The luminosity of the star (unit=solar lum). \item\end{CompactList}\item 
long double {\bf m\-Mass}\label{class_s_g___star_o1}

\begin{CompactList}\small\item\em The mass of the star (unit= solar mass). \item\end{CompactList}\item 
long double {\bf m\-Life}\label{class_s_g___star_o2}

\begin{CompactList}\small\item\em The total lifetime estimlated for the star (unit=year). \item\end{CompactList}\item 
long double {\bf m\-Age}\label{class_s_g___star_o3}

\begin{CompactList}\small\item\em The elapsed lifetile of the star (unit=year). \item\end{CompactList}\item 
long double {\bf m\-R\_\-ecosphere}\label{class_s_g___star_o4}

\begin{CompactList}\small\item\em The radius of the ecosphere (unit=UA). \item\end{CompactList}\item 
int {\bf m\-Planet\-Number}\label{class_s_g___star_o5}

\begin{CompactList}\small\item\em The number of planets orbiting around the star. \item\end{CompactList}\end{CompactItemize}
\subsection*{Protected Member Functions}
\begin{CompactItemize}
\item 
long double {\bf calculate\-Luminosity} (long double mass)
\begin{CompactList}\small\item\em Calculate the luminosity of the star, based on its mass. \item\end{CompactList}\item 
long double {\bf calculate\-Ecosphere} (long double luminosity)
\begin{CompactList}\small\item\em Calculate the ecosphere radius of the star, based on its luminosity. \item\end{CompactList}\item 
long double {\bf calculate\-Life} ()
\begin{CompactList}\small\item\em This function calculate the star life time. \item\end{CompactList}\end{CompactItemize}


\subsection{Detailed Description}
The primary star of the planetary system. 

\begin{itemize}
\item The star is initialized with the Sun characteristics.\item The star can then be defined with set\-Mass and set\-Age. Then the other parameters are automatically calculated.\item A better manual control of the star characteristics is possible with set\-Luminosity, set\-Ecosphere and set\-Life.\item The star can also be defined randomly with the set\-Random\-Star function. \end{itemize}




\subsection{Constructor \& Destructor Documentation}
\index{SG_Star@{SG\_\-Star}!SG_Star@{SG\_\-Star}}
\index{SG_Star@{SG\_\-Star}!SG_Star@{SG\_\-Star}}
\subsubsection{\setlength{\rightskip}{0pt plus 5cm}SG\_\-Star::SG\_\-Star ()}\label{class_s_g___star_a0}


Constructor. 

The star is initialized with the Sun characteristics. 

\subsection{Member Function Documentation}
\index{SG_Star@{SG\_\-Star}!calculateEcosphere@{calculateEcosphere}}
\index{calculateEcosphere@{calculateEcosphere}!SG_Star@{SG\_\-Star}}
\subsubsection{\setlength{\rightskip}{0pt plus 5cm}long double SG\_\-Star::calculate\-Ecosphere (long double {\em luminosity})\hspace{0.3cm}{\tt  [protected]}}\label{class_s_g___star_b1}


Calculate the ecosphere radius of the star, based on its luminosity. 

The ecosphere is the region of space around a star that is considered able to support life. \begin{Desc}
\item[Parameters:]
\begin{description}
\item[{\em luminosity}]The luminosity of the star (unit = solar luminosity). \end{description}
\end{Desc}
\begin{Desc}
\item[Returns:]The ecosphere radius, ie: the maximum orbit for a planet that can support life. (unit=AU). \end{Desc}
\index{SG_Star@{SG\_\-Star}!calculateLife@{calculateLife}}
\index{calculateLife@{calculateLife}!SG_Star@{SG\_\-Star}}
\subsubsection{\setlength{\rightskip}{0pt plus 5cm}long double SG\_\-Star::calculate\-Life ()\hspace{0.3cm}{\tt  [protected]}}\label{class_s_g___star_b2}


This function calculate the star life time. 

The Lifetime is based on the mass/luminosity ratio of the star. \begin{Desc}
\item[Returns:]The life duration of the star (unit = years) \end{Desc}
\begin{Desc}
\item[See also:]{\bf set\-Life}{\rm (p.\,\pageref{class_s_g___star_a6})} \end{Desc}
\index{SG_Star@{SG\_\-Star}!calculateLuminosity@{calculateLuminosity}}
\index{calculateLuminosity@{calculateLuminosity}!SG_Star@{SG\_\-Star}}
\subsubsection{\setlength{\rightskip}{0pt plus 5cm}long double SG\_\-Star::calculate\-Luminosity (long double {\em mass})\hspace{0.3cm}{\tt  [protected]}}\label{class_s_g___star_b0}


Calculate the luminosity of the star, based on its mass. 

\begin{Desc}
\item[Parameters:]
\begin{description}
\item[{\em mass}]The mass of the star (unit = solar mass) \end{description}
\end{Desc}
\begin{Desc}
\item[Returns:]The luminosity of the star (unit = solar luminosity) \end{Desc}
\index{SG_Star@{SG\_\-Star}!getBodePlanetOrbit@{getBodePlanetOrbit}}
\index{getBodePlanetOrbit@{getBodePlanetOrbit}!SG_Star@{SG\_\-Star}}
\subsubsection{\setlength{\rightskip}{0pt plus 5cm}long double SG\_\-Star::get\-Bode\-Planet\-Orbit (int {\em index})}\label{class_s_g___star_a12}


This function returns the orbit of the Nth planet of a planetary system. 

This Orbit is based on the empirical Bode-Titius Law, modified to depend on the star. This law originally is: Orbit = (3k+4)/10 AU (with k = 0,1,2,4,8,16,etc) \begin{Desc}
\item[Parameters:]
\begin{description}
\item[{\em index}]The index of the planet (0..N). Note that a negative index will return a random orbit radius. \end{description}
\end{Desc}
\begin{Desc}
\item[Returns:]The orbit of the planet (unit = AU) \end{Desc}
\index{SG_Star@{SG\_\-Star}!getBodyTemperature@{getBodyTemperature}}
\index{getBodyTemperature@{getBodyTemperature}!SG_Star@{SG\_\-Star}}
\subsubsection{\setlength{\rightskip}{0pt plus 5cm}long double SG\_\-Star::get\-Body\-Temperature (long double {\em orbit\_\-radius}, long double {\em albedo})}\label{class_s_g___star_a9}


This function estimates the surface temperature for a body (ie: a planet), at a certain distance from the star. 

This is Fogg's eq.19. \begin{Desc}
\item[Parameters:]
\begin{description}
\item[{\em orbit\_\-radius}]The distance from the star to the body (unit = AU). \item[{\em albedo}]The albedo of the body. \end{description}
\end{Desc}
\begin{Desc}
\item[Returns:]The temperature at the surface of this body (Unit = Kelvin) /$\ast$-------------------------------------------------------------------------- \end{Desc}
\index{SG_Star@{SG\_\-Star}!getEffectiveTemperature@{getEffectiveTemperature}}
\index{getEffectiveTemperature@{getEffectiveTemperature}!SG_Star@{SG\_\-Star}}
\subsubsection{\setlength{\rightskip}{0pt plus 5cm}long double SG\_\-Star::get\-Effective\-Temperature (long double {\em orbit\_\-radius}, long double {\em albedo})}\label{class_s_g___star_a10}


This function return the 'Effective Temperature' of a body. 

The effective temperature of a planet is the temperature it would have if it acted as a black body, absorbing all the incoming radiation, and reradiating all back to the space. Usually, the effective temperature is different of the real surface temperature.\begin{itemize}
\item Venus = 227 K\item Earth = 255 K\item Mars = 217 K This is Fogg's eq.19. \begin{Desc}
\item[Parameters:]
\begin{description}
\item[{\em orbit\_\-radius}]The distance from the star to the body (unit = AU). \item[{\em albedo}]The albedo of the body. \end{description}
\end{Desc}
\begin{Desc}
\item[Returns:]The Effective Temperature of this body (Unit = Kelvin) /$\ast$-------------------------------------------------------------------------- \end{Desc}
\end{itemize}
\index{SG_Star@{SG\_\-Star}!getFarthestPlanetOrbit@{getFarthestPlanetOrbit}}
\index{getFarthestPlanetOrbit@{getFarthestPlanetOrbit}!SG_Star@{SG\_\-Star}}
\subsubsection{\setlength{\rightskip}{0pt plus 5cm}long double SG\_\-Star::get\-Farthest\-Planet\-Orbit ()}\label{class_s_g___star_a13}


Renvoie l'orbite maximale o\`{u} peut se former une planete. 

\begin{Desc}
\item[Returns:]unit = UA \end{Desc}
\index{SG_Star@{SG\_\-Star}!getNearestPlanetOrbit@{getNearestPlanetOrbit}}
\index{getNearestPlanetOrbit@{getNearestPlanetOrbit}!SG_Star@{SG\_\-Star}}
\subsubsection{\setlength{\rightskip}{0pt plus 5cm}long double SG\_\-Star::get\-Nearest\-Planet\-Orbit ()}\label{class_s_g___star_a11}


Renvoie l'orbite minimale o\`{u} peut se former une planete. 

\begin{Desc}
\item[Returns:]unit = UA \end{Desc}
\index{SG_Star@{SG\_\-Star}!getStellarDustLimit@{getStellarDustLimit}}
\index{getStellarDustLimit@{getStellarDustLimit}!SG_Star@{SG\_\-Star}}
\subsubsection{\setlength{\rightskip}{0pt plus 5cm}long double SG\_\-Star::get\-Stellar\-Dust\-Limit ()}\label{class_s_g___star_a14}


This function returns the radius of the stellar dust cloud surrounding the star. 

\begin{Desc}
\item[Returns:]The stellardust cloud radius (unit = UA) \end{Desc}
\index{SG_Star@{SG\_\-Star}!setAge@{setAge}}
\index{setAge@{setAge}!SG_Star@{SG\_\-Star}}
\subsubsection{\setlength{\rightskip}{0pt plus 5cm}void SG\_\-Star::set\-Age (long double {\em age})}\label{class_s_g___star_a7}


This function set the age of the star. 

The age cannot be higher than teh life of the star. \begin{Desc}
\item[Parameters:]
\begin{description}
\item[{\em age}]The age of the star (unit = years) \end{description}
\end{Desc}
\index{SG_Star@{SG\_\-Star}!setEcosphere@{setEcosphere}}
\index{setEcosphere@{setEcosphere}!SG_Star@{SG\_\-Star}}
\subsubsection{\setlength{\rightskip}{0pt plus 5cm}void SG\_\-Star::set\-Ecosphere (long double {\em ecosphere})}\label{class_s_g___star_a5}


This function set the star ecosphere radius. 

Use this function if you want to override the calculated ecosphere by a real ecosphere radius. The ecosphere is the region of space around a star that is considered able to support life. \begin{Desc}
\item[Parameters:]
\begin{description}
\item[{\em ecosphere}]The ecosphere of the star (unit = AU) \end{description}
\end{Desc}
\begin{Desc}
\item[See also:]{\bf calculate\-Ecosphere}{\rm (p.\,\pageref{class_s_g___star_b1})} \end{Desc}
\index{SG_Star@{SG\_\-Star}!setLife@{setLife}}
\index{setLife@{setLife}!SG_Star@{SG\_\-Star}}
\subsubsection{\setlength{\rightskip}{0pt plus 5cm}void SG\_\-Star::set\-Life (long double {\em life})}\label{class_s_g___star_a6}


This function set the star life time. 

This is the life duration expected for the star, from its \char`\"{}birth\char`\"{}, to its \char`\"{}death\char`\"{}. Use this function if you want to override the calculated life by a real life time. \begin{Desc}
\item[Parameters:]
\begin{description}
\item[{\em life}]The life duration of the star (unit = years) \end{description}
\end{Desc}
\begin{Desc}
\item[See also:]{\bf calculate\-Life}{\rm (p.\,\pageref{class_s_g___star_b2})} \end{Desc}
\index{SG_Star@{SG\_\-Star}!setLuminosity@{setLuminosity}}
\index{setLuminosity@{setLuminosity}!SG_Star@{SG\_\-Star}}
\subsubsection{\setlength{\rightskip}{0pt plus 5cm}void SG\_\-Star::set\-Luminosity (long double {\em luminosity})}\label{class_s_g___star_a4}


This function set the star luminosity, and recalculate the ecosphere radius, and life. 

Use this function if you want to override the calculated luminosity by a real luminosity.

Luminosity is the total amount of energy that a star radiates into space every second. Luminosity depends on both the surface area and the surface temperature of the star, so that, for example, two stars with the same surface temperature but different luminosity must differ in size.

The luminosity of a blackbody (which most stars closely approximate) of temperature T and radius R is given by the Stefan-Boltzmann equation: L = 4 pi R178 sigma T4

where sigma is the Stefan-Boltzmann constant (5.67 x 10-8 W/m178/K4).

\begin{Desc}
\item[Parameters:]
\begin{description}
\item[{\em luminosity}]The luminosity of the star (unit = solar luminosity) \end{description}
\end{Desc}
\begin{Desc}
\item[See also:]{\bf calculate\-Luminosity}{\rm (p.\,\pageref{class_s_g___star_b0})} {\bf set\-Magnitude}{\rm (p.\,\pageref{class_s_g___star_a8})} \end{Desc}
\index{SG_Star@{SG\_\-Star}!setMagnitude@{setMagnitude}}
\index{setMagnitude@{setMagnitude}!SG_Star@{SG\_\-Star}}
\subsubsection{\setlength{\rightskip}{0pt plus 5cm}void SG\_\-Star::set\-Magnitude (long double {\em magnitude})}\label{class_s_g___star_a8}


Calculate the luminosity of the star, based on its bolometric magnitude. 

This function recalculate also the ecosphere radius and life time of the star.

Luminosity (L) is related to bolometric magnitude (Mbol) by the formula: Mbol = 2.5 log (L/Lsun) + 4.72 L = exp((Mbol -4.72)/2.5) $\ast$ Lsun Mbolsun = 4.72

The bolometric magnitude is the magnitude of a star measured across all wavelengths, so that it takes into account the total amount of energy radiated. If a star is a strong infrared or ultraviolet emitter, its bolometric magnitude will differ greatly from its visual magnitude.

\begin{Desc}
\item[Parameters:]
\begin{description}
\item[{\em magnitude}]The bolometric magnitude of the star \end{description}
\end{Desc}
\begin{Desc}
\item[See also:]{\bf calculate\-Luminosity}{\rm (p.\,\pageref{class_s_g___star_b0})} {\bf set\-Luminosity}{\rm (p.\,\pageref{class_s_g___star_a4})} \end{Desc}
\index{SG_Star@{SG\_\-Star}!setMass@{setMass}}
\index{setMass@{setMass}!SG_Star@{SG\_\-Star}}
\subsubsection{\setlength{\rightskip}{0pt plus 5cm}void SG\_\-Star::set\-Mass (long double {\em mass})}\label{class_s_g___star_a3}


This function set the star mass, and calculate its luminosity, ecosphere, and life. 

\begin{Desc}
\item[Parameters:]
\begin{description}
\item[{\em mass}]The star mass (unit = solar mass) \end{description}
\end{Desc}


The documentation for this class was generated from the following files:\begin{CompactItemize}
\item 
E:/sphinx/LFE/lib\_\-stargen/SG\_\-Star.h\item 
E:/sphinx/LFE/lib\_\-stargen/SG\_\-Star.cpp\end{CompactItemize}
