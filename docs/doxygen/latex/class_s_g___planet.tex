\section{SG\_\-Planet Class Reference}
\label{class_s_g___planet}\index{SG_Planet@{SG\_\-Planet}}
A planet (generation and properties).  


{\tt \#include $<$SG\_\-Planet.h$>$}

\subsection*{Public Member Functions}
\begin{CompactItemize}
\item 
{\bf SG\_\-Planet} ({\bf SG\_\-Star} $\ast$sun, int planet\_\-no, long double a, long double e, long double mass)\label{class_s_g___planet_a0}

\begin{CompactList}\small\item\em Constructor. \item\end{CompactList}\item 
{\bf $\sim$SG\_\-Planet} ()\label{class_s_g___planet_a1}

\begin{CompactList}\small\item\em Destructor. \item\end{CompactList}\item 
void {\bf calculate\-Planet} ()\label{class_s_g___planet_a2}

\begin{CompactList}\small\item\em This function generate the planet. \item\end{CompactList}\item 
long double {\bf get\-Orbit} ()\label{class_s_g___planet_a3}

\begin{CompactList}\small\item\em This function returns the semi-major-axis of the orbit (unit = AU). \item\end{CompactList}\item 
long double {\bf get\-Eccentricity} ()\label{class_s_g___planet_a4}

\begin{CompactList}\small\item\em This function returns the eccentricity of the orbit. \item\end{CompactList}\item 
long double {\bf get\-Mass} ()\label{class_s_g___planet_a5}

\begin{CompactList}\small\item\em This function returns the mass of the planet (unit = solar masses). \item\end{CompactList}\item 
void {\bf set\-Orbit} (long double a)\label{class_s_g___planet_a6}

\begin{CompactList}\small\item\em This function sets the semi-major-axis of the planet's orbit (unit=AU). \item\end{CompactList}\item 
void {\bf set\-Eccentricity} (long double e)\label{class_s_g___planet_a7}

\begin{CompactList}\small\item\em This function sets the eccentricity of the planet's orbit. \item\end{CompactList}\item 
void {\bf set\-Axial\-Tilt} (long double tilt)
\begin{CompactList}\small\item\em This function sets the axial inclination of the planet. \item\end{CompactList}\item 
void {\bf set\-Mass} (long double mass)
\begin{CompactList}\small\item\em This function sets the total mass of the planet (dust+gas). \item\end{CompactList}\item 
void {\bf add\-Dust\-Mass} (long double mass)
\begin{CompactList}\small\item\em This function increases the dust mass of the planet (dust, rock, and all solid elements). \item\end{CompactList}\item 
void {\bf add\-Gas\-Mass} (long double mass)
\begin{CompactList}\small\item\em This function increases the gas mass of the planet. \item\end{CompactList}\item 
void {\bf set\-Gas\-Giant} (bool giant)
\begin{CompactList}\small\item\em Set the planet as a gas giant planet. \item\end{CompactList}\item 
void {\bf set\-Planet\-Number} (int number)
\begin{CompactList}\small\item\em This function changes the number of the planet. \item\end{CompactList}\end{CompactItemize}
\subsection*{Protected Member Functions}
\begin{CompactItemize}
\item 
void {\bf calculate\-Gas\-Planet} ()
\begin{CompactList}\small\item\em Calculations for a gas planet. \item\end{CompactList}\item 
void {\bf calculate\-Solid\-Planet} ()
\begin{CompactList}\small\item\em Calculations for a solid (non-gas) planet. \item\end{CompactList}\item 
long double {\bf estimate\-Density} (long double orbit\_\-radius, long double r\_\-ecosphere, bool gas\_\-giant)
\begin{CompactList}\small\item\em This function determine the 'empirical' density of a planet. \item\end{CompactList}\item 
long double {\bf calculate\-Orbit\-Period} (long double star\_\-mass)
\begin{CompactList}\small\item\em This function returns the period of the planet orbit (Kepler Law). \item\end{CompactList}\item 
long double {\bf calculate\-Exo\-Temperature} ()
\begin{CompactList}\small\item\em This function returns the exospheric temperature of the planet. \item\end{CompactList}\item 
int {\bf calculate\-Inclination} (long double orb\_\-radius)
\begin{CompactList}\small\item\em This function determine the inclinaison angle of the planet. \item\end{CompactList}\item 
planet\_\-type {\bf calculate\-Type} (bool Gas\-Planet)
\begin{CompactList}\small\item\em This function calculates the type of the planet. \item\end{CompactList}\item 
long double {\bf calculate\-Day\-Length} ()
\begin{CompactList}\small\item\em This function calculate the length of the day on the planet. \item\end{CompactList}\item 
bool {\bf calculate\-Face\-Locking} ()
\begin{CompactList}\small\item\em This function determine if the same side of the planet is always facing the star. \item\end{CompactList}\item 
long double {\bf calculate\-Gravity} (long double distance)
\begin{CompactList}\small\item\em This function calculates the gravity of a planet. \item\end{CompactList}\item 
long double {\bf get\-Kothari\-Radius} (long double mass, bool giant, int zone)
\begin{CompactList}\small\item\em This function returns the radius of the core of the planet (units = km). \item\end{CompactList}\item 
int {\bf get\-Orbital\-Zone} (long double luminosity, long double orbit\_\-radius)
\begin{CompactList}\small\item\em This function returns the orbital zone of the planet. \item\end{CompactList}\item 
bool {\bf calculate\-Metal} ()
\begin{CompactList}\small\item\em Calculate the presence of metal in the planet. \item\end{CompactList}\item 
void {\bf calculate\-Global\-Climate} ()
\begin{CompactList}\small\item\em This function calculates all the climatic conditions on the planet. \item\end{CompactList}\item 
void {\bf estimate\-Surface\-Climate} ()\label{class_s_g___planet_b14}

\begin{CompactList}\small\item\em This function estimates the climate at the surface of the planet. \item\end{CompactList}\item 
bool {\bf calculate\-Greenhouse\-Effect} ()
\begin{CompactList}\small\item\em This function determine if the planet suffers from the Greenhouse effect. \item\end{CompactList}\item 
void {\bf calculate\-Temperature\-Range} ()\label{class_s_g___planet_b16}

\begin{CompactList}\small\item\em This function determine the min and max temperatures on the surface of the planet. \item\end{CompactList}\item 
long double {\bf calculate\-Boiling\-Point} (long double pressure)
\begin{CompactList}\small\item\em This function returns the boiling point of water on a planet, under a given atmosphere. \item\end{CompactList}\item 
long double {\bf calculate\-Hydro\-Fraction} (long double volatile\_\-gas\_\-inventory)
\begin{CompactList}\small\item\em This function calculates the fraction of the planet which is covered with water. \item\end{CompactList}\item 
long double {\bf calculate\-Ice\-Fraction} (long double hydro\_\-fraction)
\begin{CompactList}\small\item\em This function returns the fraction of the planet wich is covered with ice. \item\end{CompactList}\item 
long double {\bf calculate\-Opacity} (long double molecular\_\-weight, long double pressure)
\begin{CompactList}\small\item\em This function returns the optical depth of the atmosphere. \item\end{CompactList}\item 
long double {\bf calculate\-Albedo} (long double water\_\-fraction, long double cloud\_\-fraction, long double ice\_\-fraction)
\begin{CompactList}\small\item\em This function returns the albedo of the planet. \item\end{CompactList}\item 
long double {\bf calculate\-Cloud\-Fraction} (long double smallest\_\-MW\_\-retained, long double hydro\_\-fraction)
\begin{CompactList}\small\item\em This function return the fraction of the planet wich is covered with clouds. \item\end{CompactList}\item 
long double {\bf calculate\-Greehouse\-Rise\-Temp} (long double optical\_\-depth, long double effective\_\-temp)
\begin{CompactList}\small\item\em This function returns the rise in temperature produced by the greenhouse effect. \item\end{CompactList}\item 
void {\bf calculate\-Surface\-Climate} (long double last\_\-water, long double last\_\-clouds, long double last\_\-ice, long double last\_\-temp, long double last\_\-albedo)
\begin{CompactList}\small\item\em This function calculates the climatic conditions at the surface of the planet. \item\end{CompactList}\item 
long double {\bf calculate\-Escape\-Velocity} ()
\begin{CompactList}\small\item\em This function implements the escape velocity calculation. \item\end{CompactList}\item 
long double {\bf estimate\-Minimal\-Molecule\-Weight} ()
\begin{CompactList}\small\item\em This function returns an estimation of the smallest molecular weight retained by the planet. \item\end{CompactList}\item 
long double {\bf calculate\-Minimal\-Molecule\-Weight} ()
\begin{CompactList}\small\item\em This function returns the weight of the lightest molecule kept by the planet. \item\end{CompactList}\item 
long double {\bf calculate\-Pressure} (long double radius)
\begin{CompactList}\small\item\em This function determine the pressure on the planet. \item\end{CompactList}\item 
long double {\bf get\-RMSvelocity} (long double molecular\_\-weight)
\begin{CompactList}\small\item\em This function returns the RMS velocity of a certain gas, under the planet high atmosphere temperature. \item\end{CompactList}\item 
long double {\bf get\-Gas\-Life} (long double molecular\_\-weight)
\begin{CompactList}\small\item\em This function calculates the time it takes for 1/e of a gas to escape from a planet's atmosphere. \item\end{CompactList}\item 
long double {\bf calculate\-Volatil\-Gas\-Ratio} (long double escape\_\-vel, long double stellar\_\-mass, bool greenhouse\_\-effect, bool accreted\_\-gas)
\begin{CompactList}\small\item\em This function returns the 'inventory' of the gas in the atmosphere. \item\end{CompactList}\item 
long double {\bf soft} (long double v, long double max, long double min)\label{class_s_g___planet_b32}

\begin{CompactList}\small\item\em Function for 'soft limiting' the temperatures. \item\end{CompactList}\item 
long double {\bf get\-Sphere\-Mass} (long double density, long double radius)
\begin{CompactList}\small\item\em This function returns the mass of a sphere. \item\end{CompactList}\item 
long double {\bf get\-Sphere\-Radius} (long double mass, long double density)
\begin{CompactList}\small\item\em This function returns the radius of a sphere. \item\end{CompactList}\item 
long double {\bf get\-Sphere\-Density} (long double mass, long double radius)
\begin{CompactList}\small\item\em This function returns the density of a sphere. \item\end{CompactList}\end{CompactItemize}
\subsection*{Protected Attributes}
\begin{CompactItemize}
\item 
int {\bf m\-Planet\_\-no}\label{class_s_g___planet_p0}

\begin{CompactList}\small\item\em Planet index. \item\end{CompactList}\item 
planet\_\-type {\bf m\-Type}\label{class_s_g___planet_p1}

\begin{CompactList}\small\item\em Code for the planet's type. \item\end{CompactList}\item 
bool {\bf m\-Earthlike}\label{class_s_g___planet_p2}

\begin{CompactList}\small\item\em TRUE if the planet is very similar to Earth. \item\end{CompactList}\item 
bool {\bf m\-Gas\_\-giant}\label{class_s_g___planet_p3}

\begin{CompactList}\small\item\em TRUE if the planet is a gas giant. \item\end{CompactList}\item 
bool {\bf m\-Gas\_\-planet}\label{class_s_g___planet_p4}

\begin{CompactList}\small\item\em TRUE if the planet is a gas planet. \item\end{CompactList}\item 
long double {\bf m\_\-a}\label{class_s_g___planet_p5}

\begin{CompactList}\small\item\em Semi-major axis of the orbit (unit = AU). \item\end{CompactList}\item 
long double {\bf m\_\-e}\label{class_s_g___planet_p6}

\begin{CompactList}\small\item\em Eccentricity of the orbit (unitless). \item\end{CompactList}\item 
long double {\bf m\-Axial\_\-tilt}\label{class_s_g___planet_p7}

\begin{CompactList}\small\item\em Planet inclination (unit = degrees). \item\end{CompactList}\item 
long double {\bf m\-Mass}\label{class_s_g___planet_p8}

\begin{CompactList}\small\item\em Planet mass (unit = solar masses). \item\end{CompactList}\item 
long double {\bf m\-Dust\_\-mass}\label{class_s_g___planet_p9}

\begin{CompactList}\small\item\em Planet mass, ignoring gas (unit = solar masses). \item\end{CompactList}\item 
long double {\bf m\-Gas\_\-mass}\label{class_s_g___planet_p10}

\begin{CompactList}\small\item\em Planet mass, ignoring dust (unit = solar masses). \item\end{CompactList}\item 
long double {\bf m\-Core\_\-radius}\label{class_s_g___planet_p11}

\begin{CompactList}\small\item\em Radius of the rocky core (unit = km). \item\end{CompactList}\item 
long double {\bf m\-Radius}\label{class_s_g___planet_p12}

\begin{CompactList}\small\item\em Planet equatorial radius (unit = km). \item\end{CompactList}\item 
long double {\bf m\-Density}\label{class_s_g___planet_p13}

\begin{CompactList}\small\item\em Planet density (unit = g/cc). \item\end{CompactList}\item 
int {\bf m\-Orbit\_\-zone}\label{class_s_g___planet_p14}

\begin{CompactList}\small\item\em The 'zone' of the planet. \item\end{CompactList}\item 
long double {\bf m\-Orbit\_\-period}\label{class_s_g___planet_p15}

\begin{CompactList}\small\item\em Duration of the local year (unit = earth days). \item\end{CompactList}\item 
long double {\bf m\-Day}\label{class_s_g___planet_p16}

\begin{CompactList}\small\item\em Duration of the local day (hours). \item\end{CompactList}\item 
bool {\bf m\-Resonant\_\-period}\label{class_s_g___planet_p17}

\begin{CompactList}\small\item\em TRUE if in resonant rotation. \item\end{CompactList}\item 
bool {\bf m\-Face\-Locked}\label{class_s_g___planet_p18}

\begin{CompactList}\small\item\em The same side of the planet is always facing the star. \item\end{CompactList}\item 
long double {\bf m\-Esc\_\-velocity}\label{class_s_g___planet_p19}

\begin{CompactList}\small\item\em Escape velocity (unit = cm/sec). \item\end{CompactList}\item 
int {\bf m\-Breathe}\label{class_s_g___planet_p20}

\begin{CompactList}\small\item\em Atmosphere breathability. \item\end{CompactList}\item 
long double {\bf m\-Surf\_\-grav}\label{class_s_g___planet_p21}

\begin{CompactList}\small\item\em Surface Gravity (unit= Earth gravity). \item\end{CompactList}\item 
long double {\bf m\-Surf\_\-temp}\label{class_s_g___planet_p22}

\begin{CompactList}\small\item\em Surface temperature (unit = Kelvin). \item\end{CompactList}\item 
long double {\bf m\-Surf\_\-pressure}\label{class_s_g___planet_p23}

\begin{CompactList}\small\item\em Surface pressure (unit = millibars mb). \item\end{CompactList}\item 
long double {\bf m\-Molec\_\-weight}\label{class_s_g___planet_p24}

\begin{CompactList}\small\item\em The smallest molecular weight retained (unit=gram). \item\end{CompactList}\item 
long double {\bf m\-Boiling\_\-point}\label{class_s_g___planet_p26}

\begin{CompactList}\small\item\em The boiling point of water (Kelvin). \item\end{CompactList}\item 
long double {\bf m\-Albedo}\label{class_s_g___planet_p27}

\begin{CompactList}\small\item\em Albedo of the planet. \item\end{CompactList}\item 
long double {\bf m\-Exospheric\_\-temp}\label{class_s_g___planet_p28}

\begin{CompactList}\small\item\em Temperature (unit = Kelvin). \item\end{CompactList}\item 
bool {\bf m\-Greenhouse\_\-effect}\label{class_s_g___planet_p29}

\begin{CompactList}\small\item\em Runaway greenhouse effect ? \item\end{CompactList}\item 
long double {\bf m\-Greenhouse\_\-rise}\label{class_s_g___planet_p30}

\begin{CompactList}\small\item\em Temperature rise due to greenhouse effect. \item\end{CompactList}\item 
long double {\bf m\-High\_\-temp}\label{class_s_g___planet_p31}

\begin{CompactList}\small\item\em Day-time temperature. \item\end{CompactList}\item 
long double {\bf m\-Low\_\-temp}\label{class_s_g___planet_p32}

\begin{CompactList}\small\item\em Night-time temperature. \item\end{CompactList}\item 
long double {\bf m\-Max\_\-temp}\label{class_s_g___planet_p33}

\begin{CompactList}\small\item\em Temperature in a Summer Day (Kelvin). \item\end{CompactList}\item 
long double {\bf m\-Min\_\-temp}\label{class_s_g___planet_p34}

\begin{CompactList}\small\item\em Temperature in a Winter Night (Kelvin). \item\end{CompactList}\item 
long double {\bf m\-Hydrosphere}\label{class_s_g___planet_p35}

\begin{CompactList}\small\item\em Fraction of surface covered with water. \item\end{CompactList}\item 
long double {\bf m\-Cloud\_\-cover}\label{class_s_g___planet_p36}

\begin{CompactList}\small\item\em Fraction of surface covered with clouds. \item\end{CompactList}\item 
long double {\bf m\-Ice\_\-cover}\label{class_s_g___planet_p37}

\begin{CompactList}\small\item\em Fraction of surface covered with ice. \item\end{CompactList}\item 
bool {\bf m\-Metal}\label{class_s_g___planet_p38}

\begin{CompactList}\small\item\em Presence of metal in the planet. \item\end{CompactList}\item 
{\bf SG\_\-Star} $\ast$ {\bf m\-Primary\-Star}\label{class_s_g___planet_p39}

\begin{CompactList}\small\item\em The main Star of the solar system. \item\end{CompactList}\item 
{\bf SG\_\-Atmosphere} $\ast$ {\bf m\-Atmosphere}\label{class_s_g___planet_p40}

\begin{CompactList}\small\item\em The planet's atmosphere. \item\end{CompactList}\end{CompactItemize}
\subsection*{Friends}
\begin{CompactItemize}
\item 
class {\bf SG\_\-Atmosphere}\label{class_s_g___planet_n0}

\item 
class {\bf SG\_\-File}\label{class_s_g___planet_n1}

\end{CompactItemize}


\subsection{Detailed Description}
A planet (generation and properties). 

Planet Type is defined as follows:\begin{itemize}
\item Rock : Solid planet without atmosphere (ex: Mercure, Pluto)\item Ice : Planet mostly covered with water-ice (more than 95\%)\item Water : Planet with atmosphere, and mostly covered with water-ocean (more than 95\%)\item Terrestrial: Planet with atmosphere, and more than 5\% of water coverage (ex: Earth)\item Venusian : Planet with atmosphere, few or no water, and a very high average surface temperature\item Martian : Planet with atmosphere, few or no water, and a temperate surface temperature (ex: Mars)\item Gas\-Giant : Regular Gas planet (ex: Jupiter).\item Sub\-Gas\-Giant: Gas planet with a mass $<$ 20 EM.\item Sub\-Sub\-Gas\-Giant: Gas planet with a gas part lower than 20\% of its total mass (gas dwarf).\item Asteroids : Solid body with a very small mass. \end{itemize}




\subsection{Member Function Documentation}
\index{SG_Planet@{SG\_\-Planet}!addDustMass@{addDustMass}}
\index{addDustMass@{addDustMass}!SG_Planet@{SG\_\-Planet}}
\subsubsection{\setlength{\rightskip}{0pt plus 5cm}void SG\_\-Planet::add\-Dust\-Mass (long double {\em mass})}\label{class_s_g___planet_a10}


This function increases the dust mass of the planet (dust, rock, and all solid elements). 

\begin{Desc}
\item[Parameters:]
\begin{description}
\item[{\em mass}]Dust mass in the planet (unit = solar mass) \end{description}
\end{Desc}
\index{SG_Planet@{SG\_\-Planet}!addGasMass@{addGasMass}}
\index{addGasMass@{addGasMass}!SG_Planet@{SG\_\-Planet}}
\subsubsection{\setlength{\rightskip}{0pt plus 5cm}void SG\_\-Planet::add\-Gas\-Mass (long double {\em mass})}\label{class_s_g___planet_a11}


This function increases the gas mass of the planet. 

\begin{Desc}
\item[Parameters:]
\begin{description}
\item[{\em mass}]Gas mass in the planet (unit = solar mass) \end{description}
\end{Desc}
\index{SG_Planet@{SG\_\-Planet}!calculateAlbedo@{calculateAlbedo}}
\index{calculateAlbedo@{calculateAlbedo}!SG_Planet@{SG\_\-Planet}}
\subsubsection{\setlength{\rightskip}{0pt plus 5cm}long double SG\_\-Planet::calculate\-Albedo (long double {\em water\_\-fraction}, long double {\em cloud\_\-fraction}, long double {\em ice\_\-fraction})\hspace{0.3cm}{\tt  [protected]}}\label{class_s_g___planet_b21}


This function returns the albedo of the planet. 

The Albedo is a mesure of the light and radiation reflected by a planet.\begin{itemize}
\item Examples : snow=90\% ocean=10\% planet Earth=30\% Moon=12\%\item The cloud adjustment is the fraction of cloud cover obscuring each of the three major components of albedo that lie below the clouds. \begin{Desc}
\item[Parameters:]
\begin{description}
\item[{\em water\_\-fraction}]The fraction of the surface covered with oceans [0..1] \item[{\em cloud\_\-fraction}]The fraction of the surface covered with clouds [0..1] \item[{\em ice\_\-fraction}]The fraction of the surface covered with ice [0..1] \end{description}
\end{Desc}
\begin{Desc}
\item[Returns:]The albedo value is between 0..1 (representing 0\%..100\%) \end{Desc}
\begin{Desc}
\item[See also:]calculate\-Surface\-Temperature \end{Desc}
\end{itemize}
\index{SG_Planet@{SG\_\-Planet}!calculateBoilingPoint@{calculateBoilingPoint}}
\index{calculateBoilingPoint@{calculateBoilingPoint}!SG_Planet@{SG\_\-Planet}}
\subsubsection{\setlength{\rightskip}{0pt plus 5cm}long double SG\_\-Planet::calculate\-Boiling\-Point (long double {\em pressure})\hspace{0.3cm}{\tt  [protected]}}\label{class_s_g___planet_b17}


This function returns the boiling point of water on a planet, under a given atmosphere. 

This implements Fogg's equation 21. \begin{Desc}
\item[Parameters:]
\begin{description}
\item[{\em pressure}]The pressure submiitted to the water (Units = millibars). (Use the surface pressure to obtain the boiling temperature on the planet surface). \end{description}
\end{Desc}
\begin{Desc}
\item[Returns:]The boiling point (Units = Kelvin) \end{Desc}
\index{SG_Planet@{SG\_\-Planet}!calculateCloudFraction@{calculateCloudFraction}}
\index{calculateCloudFraction@{calculateCloudFraction}!SG_Planet@{SG\_\-Planet}}
\subsubsection{\setlength{\rightskip}{0pt plus 5cm}long double SG\_\-Planet::calculate\-Cloud\-Fraction (long double {\em smallest\_\-MW\_\-retained}, long double {\em hydro\_\-fraction})\hspace{0.3cm}{\tt  [protected]}}\label{class_s_g___planet_b22}


This function return the fraction of the planet wich is covered with clouds. 

Given the surface temperature of a planet (in Kelvin), this function returns the fraction of cloud cover available. This is Fogg's eq.23. See Hart in \char`\"{}Icarus\char`\"{} (vol 33, pp23 - 39, 1978) for an explanation. This equation is Hart's eq.3. Slightly modified by using constants and relationships from Glass's book \char`\"{}Introduction to Planetary Geology\char`\"{}, p.46. The 'CLOUD\_\-COVERAGE\_\-FACTOR' is the amount of surface area on Earth covered by one Kg. of cloud. \begin{Desc}
\item[See also:]calculate\-Surface\-Temperature \end{Desc}
\index{SG_Planet@{SG\_\-Planet}!calculateDayLength@{calculateDayLength}}
\index{calculateDayLength@{calculateDayLength}!SG_Planet@{SG\_\-Planet}}
\subsubsection{\setlength{\rightskip}{0pt plus 5cm}long double SG\_\-Planet::calculate\-Day\-Length ()\hspace{0.3cm}{\tt  [protected]}}\label{class_s_g___planet_b7}


This function calculate the length of the day on the planet. 

Fogg's information for this routine came from Dole \char`\"{}Habitable Planets for Man\char`\"{}, Blaisdell Publishing Company, NY, 1964. From this, he came up with his eq.12, which is the equation for the 'base\_\-angular\_\-velocity' below. He then used an equation for the change in angular velocity per time (dw/dt) from P. Goldreich and S. Soter's paper \char`\"{}Q in the Solar System\char`\"{} in Icarus, vol 5, pp.375-389 (1966). Using as a comparison the change in angular velocity for the Earth, Fogg has come up with an approximation for our new planet (his eq.13) and take that into account. This is used to find 'change\_\-in\_\-angular\_\-velocity' below.

Used parameters are mass (in solar masses), radius (in Km), orbital period (in days), orbital radius (in AU), density (in g/cc), eccentricity, and whether it is a gas giant or not.

Note: OK when applied on the solar system planets, but wrong (16h) when applied to the Earth. Maybe because the Moon influence is not taken into account. \begin{Desc}
\item[Returns:]The length of the day (units = hours). \end{Desc}
\index{SG_Planet@{SG\_\-Planet}!calculateEscapeVelocity@{calculateEscapeVelocity}}
\index{calculateEscapeVelocity@{calculateEscapeVelocity}!SG_Planet@{SG\_\-Planet}}
\subsubsection{\setlength{\rightskip}{0pt plus 5cm}long double SG\_\-Planet::calculate\-Escape\-Velocity ()\hspace{0.3cm}{\tt  [protected]}}\label{class_s_g___planet_b25}


This function implements the escape velocity calculation. 

In a given gravitational field, the escape velocity is the minimum speed an object needs to have to move away indefinitely from the planet, as opposed to falling back or staying in an orbit. Examples: earth=11km/s moon=2km/s sun=617km/s Note that it appears that Fogg's eq.15 is incorrect. \begin{Desc}
\item[Returns:]The escape velocity of the planet (unit = cm/sec).\end{Desc}


The mass is in units of solar mass.

The radius is in kilometers. \index{SG_Planet@{SG\_\-Planet}!calculateExoTemperature@{calculateExoTemperature}}
\index{calculateExoTemperature@{calculateExoTemperature}!SG_Planet@{SG\_\-Planet}}
\subsubsection{\setlength{\rightskip}{0pt plus 5cm}long double SG\_\-Planet::calculate\-Exo\-Temperature ()\hspace{0.3cm}{\tt  [protected]}}\label{class_s_g___planet_b4}


This function returns the exospheric temperature of the planet. 

The exosphere is the outermost, least dense portion of the atmosphere. In this area, the gas molecule velocity increase (especially H and He), and the molecules may escape from the atmosphere. For this reason, the temperature also increase in the exosphere. For the Earth, the exospheric temperature is about 1000 C. \begin{Desc}
\item[Returns:]The exospheric temperature of the planet (unit = Kelvin). \end{Desc}
\index{SG_Planet@{SG\_\-Planet}!calculateFaceLocking@{calculateFaceLocking}}
\index{calculateFaceLocking@{calculateFaceLocking}!SG_Planet@{SG\_\-Planet}}
\subsubsection{\setlength{\rightskip}{0pt plus 5cm}bool SG\_\-Planet::calculate\-Face\-Locking ()\hspace{0.3cm}{\tt  [protected]}}\label{class_s_g___planet_b8}


This function determine if the same side of the planet is always facing the star. 

\begin{Desc}
\item[Returns:]TRUE if a face of the planet is locked toward the star. \end{Desc}
\index{SG_Planet@{SG\_\-Planet}!calculateGasPlanet@{calculateGasPlanet}}
\index{calculateGasPlanet@{calculateGasPlanet}!SG_Planet@{SG\_\-Planet}}
\subsubsection{\setlength{\rightskip}{0pt plus 5cm}void SG\_\-Planet::calculate\-Gas\-Planet ()\hspace{0.3cm}{\tt  [protected]}}\label{class_s_g___planet_b0}


Calculations for a gas planet. 

Note that the calculation function must be called in this order, because they are using values calculated by the calls of previous functions.\begin{itemize}
\item Gas planet have no accessible surface, or no surface at all.\item We call 'surface temperature' the temperature at 1 bar.\item (Jupiter=165K - Saturn=135K - Neptune=75K - Uranus=70K)\item We call 'radius' the limit where the gas pressure is 0.1 bar. \end{itemize}
\index{SG_Planet@{SG\_\-Planet}!calculateGlobalClimate@{calculateGlobalClimate}}
\index{calculateGlobalClimate@{calculateGlobalClimate}!SG_Planet@{SG\_\-Planet}}
\subsubsection{\setlength{\rightskip}{0pt plus 5cm}void SG\_\-Planet::calculate\-Global\-Climate ()\hspace{0.3cm}{\tt  [protected]}}\label{class_s_g___planet_b13}


This function calculates all the climatic conditions on the planet. 

This function calls calculate\-Surface\-Climate until converging to 1/4 degree. \begin{Desc}
\item[See also:]{\bf calculate\-Solid\-Planet}{\rm (p.\,\pageref{class_s_g___planet_b1})} {\bf calculate\-Surface\-Climate}{\rm (p.\,\pageref{class_s_g___planet_b24})} \end{Desc}
\index{SG_Planet@{SG\_\-Planet}!calculateGravity@{calculateGravity}}
\index{calculateGravity@{calculateGravity}!SG_Planet@{SG\_\-Planet}}
\subsubsection{\setlength{\rightskip}{0pt plus 5cm}long double SG\_\-Planet::calculate\-Gravity (long double {\em distance})\hspace{0.3cm}{\tt  [protected]}}\label{class_s_g___planet_b9}


This function calculates the gravity of a planet. 

\begin{Desc}
\item[Parameters:]
\begin{description}
\item[{\em distance}]The distance from the planet center, where the gravity is calculated. (units = km) (use m\-Radius to obtain the surface gravity). \end{description}
\end{Desc}
\begin{Desc}
\item[Returns:]The gravity (Units = Earth gravities)\end{Desc}


Acceleration (units = cm/sec2) \index{SG_Planet@{SG\_\-Planet}!calculateGreehouseRiseTemp@{calculateGreehouseRiseTemp}}
\index{calculateGreehouseRiseTemp@{calculateGreehouseRiseTemp}!SG_Planet@{SG\_\-Planet}}
\subsubsection{\setlength{\rightskip}{0pt plus 5cm}long double SG\_\-Planet::calculate\-Greehouse\-Rise\-Temp (long double {\em optical\_\-depth}, long double {\em effective\_\-temp})\hspace{0.3cm}{\tt  [protected]}}\label{class_s_g___planet_b23}


This function returns the rise in temperature produced by the greenhouse effect. 

The Greenhouse effect is the warming trend on the surface, and on the lower atmosphere of a planet, that occcurs when solar radiation is trapped, as emissions from the planet. Like in a greehouse, the atmosphere may keep the solar radiation, and rise the surface temperature.\begin{itemize}
\item This is Fogg's eq.20, and is also Hart's eq.20 in his \char`\"{}Evolution of Earth's Atmosphere\char`\"{} article.\item Note: pow(x,0.25) was changed to pow(x,0.4) to match Venus.\item Examples: Earth = +30176 Venus = +500176\end{itemize}


\begin{Desc}
\item[Returns:]The temperature rise, due to the Gree\-House effect (units = Kelvin). \end{Desc}
\index{SG_Planet@{SG\_\-Planet}!calculateGreenhouseEffect@{calculateGreenhouseEffect}}
\index{calculateGreenhouseEffect@{calculateGreenhouseEffect}!SG_Planet@{SG\_\-Planet}}
\subsubsection{\setlength{\rightskip}{0pt plus 5cm}bool SG\_\-Planet::calculate\-Greenhouse\-Effect ()\hspace{0.3cm}{\tt  [protected]}}\label{class_s_g___planet_b15}


This function determine if the planet suffers from the Greenhouse effect. 

The new definition is based on the inital surface temperature and what state water is in. If it's too hot, the water will never condense out of the atmosphere, rain down and form an ocean. The albedo used here was chosen so that the boundary is about the same as the old method Neither zone, nor r\_\-greenhouse are used in this version Note: If the orbital radius of the planet is too big, 99\% of it's volatiles are assumed to have been deposited in surface reservoirs, (otherwise, it suffers from the greenhouse effect). \index{SG_Planet@{SG\_\-Planet}!calculateHydroFraction@{calculateHydroFraction}}
\index{calculateHydroFraction@{calculateHydroFraction}!SG_Planet@{SG\_\-Planet}}
\subsubsection{\setlength{\rightskip}{0pt plus 5cm}long double SG\_\-Planet::calculate\-Hydro\-Fraction (long double {\em volatile\_\-gas\_\-inventory})\hspace{0.3cm}{\tt  [protected]}}\label{class_s_g___planet_b18}


This function calculates the fraction of the planet which is covered with water. 

This function is Fogg's eq.22. Given the volatile gas inventory and planetary radius of a planet (in Km), this function returns the fraction of the planet covered with water. Note: the fraction of Earth's surface covered by water is 71\%, not 75\% as Fogg used. \begin{Desc}
\item[See also:]calculate\-Surface\-Temperature \end{Desc}
\index{SG_Planet@{SG\_\-Planet}!calculateIceFraction@{calculateIceFraction}}
\index{calculateIceFraction@{calculateIceFraction}!SG_Planet@{SG\_\-Planet}}
\subsubsection{\setlength{\rightskip}{0pt plus 5cm}long double SG\_\-Planet::calculate\-Ice\-Fraction (long double {\em hydro\_\-fraction})\hspace{0.3cm}{\tt  [protected]}}\label{class_s_g___planet_b19}


This function returns the fraction of the planet wich is covered with ice. 

Given the surface temperature of a planet (in Kelvin), this function returns the fraction of the planet's surface covered by ice.\begin{itemize}
\item This is Fogg's eq.24. See Hart[24] in Icarus vol.33, p.28 for an explanation.\item Modify constant from 70 to 90 in order to bring it more in line with the fraction of the Earth's surface covered with ice, which is approximatly 1.6\%. \begin{Desc}
\item[See also:]calculate\-Surface\-Temperature \end{Desc}
\end{itemize}
\index{SG_Planet@{SG\_\-Planet}!calculateInclination@{calculateInclination}}
\index{calculateInclination@{calculateInclination}!SG_Planet@{SG\_\-Planet}}
\subsubsection{\setlength{\rightskip}{0pt plus 5cm}int SG\_\-Planet::calculate\-Inclination (long double {\em orb\_\-radius})\hspace{0.3cm}{\tt  [protected]}}\label{class_s_g___planet_b5}


This function determine the inclinaison angle of the planet. 

If the Axial tilt is already set, we don't change it. \begin{Desc}
\item[Parameters:]
\begin{description}
\item[{\em orb\_\-radius}]The orbital radius (Unit = Astronomical Units AU) \end{description}
\end{Desc}
\begin{Desc}
\item[Returns:]The planet inclination (Unit = degrees) \end{Desc}
\index{SG_Planet@{SG\_\-Planet}!calculateMetal@{calculateMetal}}
\index{calculateMetal@{calculateMetal}!SG_Planet@{SG\_\-Planet}}
\subsubsection{\setlength{\rightskip}{0pt plus 5cm}bool SG\_\-Planet::calculate\-Metal ()\hspace{0.3cm}{\tt  [protected]}}\label{class_s_g___planet_b12}


Calculate the presence of metal in the planet. 

This function is based on the planet density and the molecules retained. It can be improved. \index{SG_Planet@{SG\_\-Planet}!calculateMinimalMoleculeWeight@{calculateMinimalMoleculeWeight}}
\index{calculateMinimalMoleculeWeight@{calculateMinimalMoleculeWeight}!SG_Planet@{SG\_\-Planet}}
\subsubsection{\setlength{\rightskip}{0pt plus 5cm}long double SG\_\-Planet::calculate\-Minimal\-Molecule\-Weight ()\hspace{0.3cm}{\tt  [protected]}}\label{class_s_g___planet_b27}


This function returns the weight of the lightest molecule kept by the planet. 

\begin{Desc}
\item[Returns:]The exact minimal molecule weight, on this planet. \end{Desc}
\begin{Desc}
\item[See also:]{\bf estimate\-Minimal\-Molecule\-Weight}{\rm (p.\,\pageref{class_s_g___planet_b26})} \end{Desc}
\index{SG_Planet@{SG\_\-Planet}!calculateOpacity@{calculateOpacity}}
\index{calculateOpacity@{calculateOpacity}!SG_Planet@{SG\_\-Planet}}
\subsubsection{\setlength{\rightskip}{0pt plus 5cm}long double SG\_\-Planet::calculate\-Opacity (long double {\em molecular\_\-weight}, long double {\em pressure})\hspace{0.3cm}{\tt  [protected]}}\label{class_s_g___planet_b20}


This function returns the optical depth of the atmosphere. 

The optical depth increase when the atmosphere is composed with light elements and when the atmosphere is thin (low pressure). The optical depth is useful in determining the amount of greenhouse effect on a planet. \begin{Desc}
\item[Parameters:]
\begin{description}
\item[{\em molecular\_\-weight}]The atmosphere melecule weight (unit = gram) \item[{\em pressure}]The atmosphere pressure at the surface of the planet (unit=m\-B) \end{description}
\end{Desc}
\begin{Desc}
\item[Returns:]The Optical Depth (unitless) \end{Desc}
\index{SG_Planet@{SG\_\-Planet}!calculateOrbitPeriod@{calculateOrbitPeriod}}
\index{calculateOrbitPeriod@{calculateOrbitPeriod}!SG_Planet@{SG\_\-Planet}}
\subsubsection{\setlength{\rightskip}{0pt plus 5cm}long double SG\_\-Planet::calculate\-Orbit\-Period (long double {\em star\_\-mass})\hspace{0.3cm}{\tt  [protected]}}\label{class_s_g___planet_b3}


This function returns the period of the planet orbit (Kepler Law). 

\begin{Desc}
\item[Parameters:]
\begin{description}
\item[{\em star\_\-mass}]Mass of the primary star (unit = Solar Mass) \end{description}
\end{Desc}
\begin{Desc}
\item[Returns:]the orbit period of the planet (unit = Earth days) \end{Desc}
\index{SG_Planet@{SG\_\-Planet}!calculatePressure@{calculatePressure}}
\index{calculatePressure@{calculatePressure}!SG_Planet@{SG\_\-Planet}}
\subsubsection{\setlength{\rightskip}{0pt plus 5cm}long double SG\_\-Planet::calculate\-Pressure (long double {\em radius})\hspace{0.3cm}{\tt  [protected]}}\label{class_s_g___planet_b28}


This function determine the pressure on the planet. 

This implements Fogg's equation 18. It use the 'inventory' of the atmosphere (reference value = 1000). This is more dedicated for solid (non gas) planets. \begin{Desc}
\item[Parameters:]
\begin{description}
\item[{\em radius}]The distance from the planet center where the pressure is calculated. (unit=km) (use m\-Radius to get the planet surface pressure). \end{description}
\end{Desc}
\begin{Desc}
\item[Returns:]The pressure (Units = millibars) \end{Desc}
\index{SG_Planet@{SG\_\-Planet}!calculateSolidPlanet@{calculateSolidPlanet}}
\index{calculateSolidPlanet@{calculateSolidPlanet}!SG_Planet@{SG\_\-Planet}}
\subsubsection{\setlength{\rightskip}{0pt plus 5cm}void SG\_\-Planet::calculate\-Solid\-Planet ()\hspace{0.3cm}{\tt  [protected]}}\label{class_s_g___planet_b1}


Calculations for a solid (non-gas) planet. 

Note that the calculation function must be called in this order, because they are using values calculated by the calls of previous functions. \index{SG_Planet@{SG\_\-Planet}!calculateSurfaceClimate@{calculateSurfaceClimate}}
\index{calculateSurfaceClimate@{calculateSurfaceClimate}!SG_Planet@{SG\_\-Planet}}
\subsubsection{\setlength{\rightskip}{0pt plus 5cm}void SG\_\-Planet::calculate\-Surface\-Climate (long double {\em last\_\-water}, long double {\em last\_\-clouds}, long double {\em last\_\-ice}, long double {\em last\_\-temp}, long double {\em last\_\-albedo})\hspace{0.3cm}{\tt  [protected]}}\label{class_s_g___planet_b24}


This function calculates the climatic conditions at the surface of the planet. 

This function estimate the ice, water, cloud fractions at the surface of the planet and the minimal, average and maximal temperatures. \index{SG_Planet@{SG\_\-Planet}!calculateType@{calculateType}}
\index{calculateType@{calculateType}!SG_Planet@{SG\_\-Planet}}
\subsubsection{\setlength{\rightskip}{0pt plus 5cm}SG\_\-Planet::planet\_\-type SG\_\-Planet::calculate\-Type (bool {\em Gas\-Planet})\hspace{0.3cm}{\tt  [protected]}}\label{class_s_g___planet_b6}


This function calculates the type of the planet. 

\begin{Desc}
\item[Parameters:]
\begin{description}
\item[{\em Gas\-Planet}]Set to TRUE, if you want to indicate that the planet is a gas planet. \end{description}
\end{Desc}
\begin{Desc}
\item[Returns:]The type of the planet (see planet\_\-type) \end{Desc}
\index{SG_Planet@{SG\_\-Planet}!calculateVolatilGasRatio@{calculateVolatilGasRatio}}
\index{calculateVolatilGasRatio@{calculateVolatilGasRatio}!SG_Planet@{SG\_\-Planet}}
\subsubsection{\setlength{\rightskip}{0pt plus 5cm}long double SG\_\-Planet::calculate\-Volatil\-Gas\-Ratio (long double {\em escape\_\-vel}, long double {\em stellar\_\-mass}, bool {\em greenhouse\_\-effect}, bool {\em accreted\_\-gas})\hspace{0.3cm}{\tt  [protected]}}\label{class_s_g___planet_b31}


This function returns the 'inventory' of the gas in the atmosphere. 

This function implements Fogg's equation 17. The Earth inventory is 1000 (reference). \begin{Desc}
\item[Parameters:]
\begin{description}
\item[{\em escape\_\-vel}]The escape velocity of the planet (unit=cm/sec). \item[{\em stellar\_\-mass}]The mass of the star (unit=solar mass) \item[{\em greenhouse\_\-effect}]TRUE if the planet is under a greenhouse effect. \item[{\em accreted\_\-gas}]TRUE if the planet has a certain amount of gas. \end{description}
\end{Desc}
\begin{Desc}
\item[Returns:]The inventory (unitless). \end{Desc}
\begin{Desc}
\item[See also:]calculate\-Surface\-Temperature generate\-Planet \end{Desc}
\index{SG_Planet@{SG\_\-Planet}!estimateDensity@{estimateDensity}}
\index{estimateDensity@{estimateDensity}!SG_Planet@{SG\_\-Planet}}
\subsubsection{\setlength{\rightskip}{0pt plus 5cm}long double SG\_\-Planet::estimate\-Density (long double {\em orbit\_\-radius}, long double {\em r\_\-ecosphere}, bool {\em gas\_\-giant})\hspace{0.3cm}{\tt  [protected]}}\label{class_s_g___planet_b2}


This function determine the 'empirical' density of a planet. 

\begin{Desc}
\item[Parameters:]
\begin{description}
\item[{\em orbit\_\-radius}]The orbital radius (Units = AU). \item[{\em r\_\-ecosphere}]The radius of the Star ecosphere (Units = AU). \item[{\em gas\_\-giant}]TRUE if you want to estimate for a gaz geant planet. \end{description}
\end{Desc}
\begin{Desc}
\item[Returns:]The 'empirical' density (Units = grams/cc) /$\ast$-------------------------------------------------------------------------- \end{Desc}
\index{SG_Planet@{SG\_\-Planet}!estimateMinimalMoleculeWeight@{estimateMinimalMoleculeWeight}}
\index{estimateMinimalMoleculeWeight@{estimateMinimalMoleculeWeight}!SG_Planet@{SG\_\-Planet}}
\subsubsection{\setlength{\rightskip}{0pt plus 5cm}long double SG\_\-Planet::estimate\-Minimal\-Molecule\-Weight ()\hspace{0.3cm}{\tt  [protected]}}\label{class_s_g___planet_b26}


This function returns an estimation of the smallest molecular weight retained by the planet. 

This is an approximate value (molecule\_\-limit), which is useful for determining the atmosphere composition. This function is based on the 'cinetic gas theory'.\begin{itemize}
\item PV = n\-RT (perfect gas equation)\item E = mv178/2 (cinetic energy)\item RT = mv178/3 \begin{Desc}
\item[Returns:]The minimal molecule weight retained by the planet gravity. (unit=gram) \end{Desc}
\begin{Desc}
\item[See also:]{\bf calculate\-Minimal\-Molecule\-Weight}{\rm (p.\,\pageref{class_s_g___planet_b27})} \end{Desc}
\end{itemize}
\index{SG_Planet@{SG\_\-Planet}!getGasLife@{getGasLife}}
\index{getGasLife@{getGasLife}!SG_Planet@{SG\_\-Planet}}
\subsubsection{\setlength{\rightskip}{0pt plus 5cm}long double SG\_\-Planet::get\-Gas\-Life (long double {\em molecular\_\-weight})\hspace{0.3cm}{\tt  [protected]}}\label{class_s_g___planet_b30}


This function calculates the time it takes for 1/e of a gas to escape from a planet's atmosphere. 

Taken from Dole p.34. He cites Jeans (1916) \& Jones (1923) \begin{Desc}
\item[Parameters:]
\begin{description}
\item[{\em molecular\_\-weight}]\end{description}
\end{Desc}
\begin{Desc}
\item[Returns:]Gas life time (unit = years) \end{Desc}
\index{SG_Planet@{SG\_\-Planet}!getKothariRadius@{getKothariRadius}}
\index{getKothariRadius@{getKothariRadius}!SG_Planet@{SG\_\-Planet}}
\subsubsection{\setlength{\rightskip}{0pt plus 5cm}long double SG\_\-Planet::get\-Kothari\-Radius (long double {\em mass}, bool {\em giant}, int {\em zone})\hspace{0.3cm}{\tt  [protected]}}\label{class_s_g___planet_b10}


This function returns the radius of the core of the planet (units = km). 

This formula is listed as eq.9 in Fogg's article, although some typos crop up in that eq. See \char`\"{}The Internal Constitution of Planets\char`\"{}, by Dr. D. S. Kothari, Mon. Not. of the Royal Astronomical Society, vol 96 pp.833-843, 1936 for the derivation. Specifically, this is Kothari's eq.23, which appears on page 840. \index{SG_Planet@{SG\_\-Planet}!getOrbitalZone@{getOrbitalZone}}
\index{getOrbitalZone@{getOrbitalZone}!SG_Planet@{SG\_\-Planet}}
\subsubsection{\setlength{\rightskip}{0pt plus 5cm}int SG\_\-Planet::get\-Orbital\-Zone (long double {\em luminosity}, long double {\em orbit\_\-radius})\hspace{0.3cm}{\tt  [protected]}}\label{class_s_g___planet_b11}


This function returns the orbital zone of the planet. 

Zone 1: Zone near of the sun, luminous and warm (ie: Earth). Zone 2: Medium zone. Zone 3: Zone far from the sun. The sun looks like a star from there. \begin{Desc}
\item[Parameters:]
\begin{description}
\item[{\em luminosity}]The lumionosity of the sun (ie: the primary star). \item[{\em orbit\_\-radius}]The radius of the orbit of the planet (units = UA) \end{description}
\end{Desc}
\begin{Desc}
\item[Returns:]The 'orbital zone' of the planet (1-2-3). \end{Desc}
\index{SG_Planet@{SG\_\-Planet}!getRMSvelocity@{getRMSvelocity}}
\index{getRMSvelocity@{getRMSvelocity}!SG_Planet@{SG\_\-Planet}}
\subsubsection{\setlength{\rightskip}{0pt plus 5cm}long double SG\_\-Planet::get\-RMSvelocity (long double {\em molecular\_\-weight})\hspace{0.3cm}{\tt  [protected]}}\label{class_s_g___planet_b29}


This function returns the RMS velocity of a certain gas, under the planet high atmosphere temperature. 

This is Fogg's eq.16. The molecular weight is used as the basis of the Root Mean Square (RMS) velocity of the molecule or atom. \begin{Desc}
\item[Parameters:]
\begin{description}
\item[{\em molecular\_\-weight}]The molecular weight of the gas (unit=gram) \end{description}
\end{Desc}
\begin{Desc}
\item[Returns:]The RMS velocity of the gas molecules (unit=cm/sec) \end{Desc}
\index{SG_Planet@{SG\_\-Planet}!getSphereDensity@{getSphereDensity}}
\index{getSphereDensity@{getSphereDensity}!SG_Planet@{SG\_\-Planet}}
\subsubsection{\setlength{\rightskip}{0pt plus 5cm}long double SG\_\-Planet::get\-Sphere\-Density (long double {\em mass}, long double {\em radius})\hspace{0.3cm}{\tt  [protected]}}\label{class_s_g___planet_b35}


This function returns the density of a sphere. 

\begin{Desc}
\item[Parameters:]
\begin{description}
\item[{\em mass}]The mass of the sphere (unit = solar masses). \item[{\em radius}]The equatorial radius of the sphere (unit = km). \end{description}
\end{Desc}
\begin{Desc}
\item[Returns:]The density of the sphere (unit = gram/cm3). \end{Desc}
\index{SG_Planet@{SG\_\-Planet}!getSphereMass@{getSphereMass}}
\index{getSphereMass@{getSphereMass}!SG_Planet@{SG\_\-Planet}}
\subsubsection{\setlength{\rightskip}{0pt plus 5cm}long double SG\_\-Planet::get\-Sphere\-Mass (long double {\em density}, long double {\em radius})\hspace{0.3cm}{\tt  [protected]}}\label{class_s_g___planet_b33}


This function returns the mass of a sphere. 

\begin{Desc}
\item[Parameters:]
\begin{description}
\item[{\em density}]The density of the sphere (units = gram/cm3) \item[{\em radius}]The radius of the sphere (units = km) \end{description}
\end{Desc}
\begin{Desc}
\item[Returns:]The mass of the sphere (units = solar masses) \end{Desc}
\index{SG_Planet@{SG\_\-Planet}!getSphereRadius@{getSphereRadius}}
\index{getSphereRadius@{getSphereRadius}!SG_Planet@{SG\_\-Planet}}
\subsubsection{\setlength{\rightskip}{0pt plus 5cm}long double SG\_\-Planet::get\-Sphere\-Radius (long double {\em mass}, long double {\em density})\hspace{0.3cm}{\tt  [protected]}}\label{class_s_g___planet_b34}


This function returns the radius of a sphere. 

\begin{Desc}
\item[Parameters:]
\begin{description}
\item[{\em mass}]The mass of the sphere (units = solar masses) \item[{\em density}]The density of the sphere (units = gram/cm3) \end{description}
\end{Desc}
\begin{Desc}
\item[Returns:]The radius of the sphere (units = km) \end{Desc}
\index{SG_Planet@{SG\_\-Planet}!setAxialTilt@{setAxialTilt}}
\index{setAxialTilt@{setAxialTilt}!SG_Planet@{SG\_\-Planet}}
\subsubsection{\setlength{\rightskip}{0pt plus 5cm}void SG\_\-Planet::set\-Axial\-Tilt (long double {\em tilt})}\label{class_s_g___planet_a8}


This function sets the axial inclination of the planet. 

\begin{Desc}
\item[Parameters:]
\begin{description}
\item[{\em tilt}]Planet inclination (unit = degree) \end{description}
\end{Desc}
\index{SG_Planet@{SG\_\-Planet}!setGasGiant@{setGasGiant}}
\index{setGasGiant@{setGasGiant}!SG_Planet@{SG\_\-Planet}}
\subsubsection{\setlength{\rightskip}{0pt plus 5cm}void SG\_\-Planet::set\-Gas\-Giant (bool {\em giant})}\label{class_s_g___planet_a12}


Set the planet as a gas giant planet. 

\begin{Desc}
\item[Parameters:]
\begin{description}
\item[{\em giant}]TRUE if the planet is a gas giant planet \end{description}
\end{Desc}
\index{SG_Planet@{SG\_\-Planet}!setMass@{setMass}}
\index{setMass@{setMass}!SG_Planet@{SG\_\-Planet}}
\subsubsection{\setlength{\rightskip}{0pt plus 5cm}void SG\_\-Planet::set\-Mass (long double {\em mass})}\label{class_s_g___planet_a9}


This function sets the total mass of the planet (dust+gas). 

\begin{Desc}
\item[Parameters:]
\begin{description}
\item[{\em mass}]Planet mass (unit = solar mass) \end{description}
\end{Desc}
\index{SG_Planet@{SG\_\-Planet}!setPlanetNumber@{setPlanetNumber}}
\index{setPlanetNumber@{setPlanetNumber}!SG_Planet@{SG\_\-Planet}}
\subsubsection{\setlength{\rightskip}{0pt plus 5cm}void SG\_\-Planet::set\-Planet\-Number (int {\em number})}\label{class_s_g___planet_a13}


This function changes the number of the planet. 

At the begining of the process, the number is the creation order of the planets, but during the sorting process, the number is set to the orbital order of the planet. \begin{Desc}
\item[Parameters:]
\begin{description}
\item[{\em number}]The number is in the range [0..N] \end{description}
\end{Desc}


The documentation for this class was generated from the following files:\begin{CompactItemize}
\item 
E:/sphinx/LFE/lib\_\-stargen/SG\_\-Planet.h\item 
E:/sphinx/LFE/lib\_\-stargen/SG\_\-Planet.cpp\end{CompactItemize}
