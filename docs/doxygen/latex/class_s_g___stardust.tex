\section{SG\_\-Stardust Class Reference}
\label{class_s_g___stardust}\index{SG_Stardust@{SG\_\-Stardust}}
The cloud of dusts and particles, which is around the star. (Accrete).  


{\tt \#include $<$SG\_\-Stardust.h$>$}

\subsection*{Public Member Functions}
\begin{CompactItemize}
\item 
{\bf SG\_\-Stardust} ({\bf SG\_\-Star} $\ast$star, int max\_\-planet=10)
\begin{CompactList}\small\item\em Constructor. \item\end{CompactList}\item 
{\bf $\sim$SG\_\-Stardust} ()\label{class_s_g___stardust_a1}

\begin{CompactList}\small\item\em Destructor. \item\end{CompactList}\item 
long double {\bf get\-Dust\-Density\-Coeff} ()\label{class_s_g___stardust_a2}

\begin{CompactList}\small\item\em This function returns the coefficient of density of the dust in the cloud. \item\end{CompactList}\item 
void {\bf set\-Dust\-Density\-Ratio} (long double ratio=1)
\begin{CompactList}\small\item\em This function changes the coefficient of density of the dust inthe cloud. \item\end{CompactList}\item 
void {\bf generate\-Planets} ()\label{class_s_g___stardust_a4}

\begin{CompactList}\small\item\em This function accrete the stellardust into several planets. \item\end{CompactList}\item 
{\bf SG\_\-Planet} $\ast$ {\bf get\-Planet} (int index)
\begin{CompactList}\small\item\em This function returns a planet (sorted by orbit radius). \item\end{CompactList}\end{CompactItemize}
\subsection*{Protected Member Functions}
\begin{CompactItemize}
\item 
bool {\bf is\-Dust\-Available} (long double inside\_\-range, long double outside\_\-range)
\begin{CompactList}\small\item\em This function indicates if there is still some stellar dust in the specified orbital range. \item\end{CompactList}\item 
long double {\bf get\-Inner\-Effect\-Limit} (proto\-Planet $\ast$protoplanet)
\begin{CompactList}\small\item\em This function returns the inner orbit affected by this proto planet. \item\end{CompactList}\item 
long double {\bf get\-Outer\-Effect\-Limit} (proto\-Planet $\ast$protoplanet)
\begin{CompactList}\small\item\em This function returns the outer orbit affected by this proto planet. \item\end{CompactList}\item 
long double {\bf get\-Critical\-Mass} (proto\-Planet $\ast$protoplanet)
\begin{CompactList}\small\item\em This function returns the mass at which a planet at this orbit will begin to accrete gas and dust. \item\end{CompactList}\item 
void {\bf accrete\-Dust} (proto\-Planet $\ast$protoplanet)
\begin{CompactList}\small\item\em This function accrete the dust located on the protoplanet orbital path. \item\end{CompactList}\item 
void {\bf update\-Dust\-Lanes} (proto\-Planet $\ast$protoplanet)
\begin{CompactList}\small\item\em This function updates the dustbands surrounding the star. \item\end{CompactList}\item 
void {\bf coalesce\-Protoplanets} (proto\-Planet $\ast$protoplanet)\label{class_s_g___stardust_b6}

\begin{CompactList}\small\item\em This function aggregates the protoplanet with the other bodies from the close orbits, and eventually create a planet. \item\end{CompactList}\item 
void {\bf collect\-Dust} (proto\-Planet $\ast$protoplanet, dust\_\-band $\ast$dustband)
\begin{CompactList}\small\item\em This function collects Dust from all the existing dustbands, starting at the given dustband. \item\end{CompactList}\item 
void {\bf sort\-Planet\-Table} ()\label{class_s_g___stardust_b8}

\begin{CompactList}\small\item\em This function sorts the array of planets, following the increasing orbit radius. \item\end{CompactList}\item 
void {\bf swap\-Planet} (int index1, int index2)\label{class_s_g___stardust_b9}

\begin{CompactList}\small\item\em This function swaps two planets in the planet array. \item\end{CompactList}\end{CompactItemize}
\subsection*{Protected Attributes}
\begin{CompactItemize}
\item 
bool {\bf m\-Dust\_\-left}\label{class_s_g___stardust_p0}

\begin{CompactList}\small\item\em TRUE if there is still dust in the cloud. \item\end{CompactList}\item 
long double {\bf m\-Dust\_\-inner\_\-limit}\label{class_s_g___stardust_p1}

\begin{CompactList}\small\item\em Inner limit of the stardust cloud. \item\end{CompactList}\item 
long double {\bf m\-Dust\_\-outer\_\-limit}\label{class_s_g___stardust_p2}

\begin{CompactList}\small\item\em Outer limit of the stardust cloud. \item\end{CompactList}\item 
long double {\bf m\-Planet\_\-inner\_\-bound}\label{class_s_g___stardust_p3}

\begin{CompactList}\small\item\em Inner limit of the planet formation area. \item\end{CompactList}\item 
long double {\bf m\-Planet\_\-outer\_\-bound}\label{class_s_g___stardust_p4}

\begin{CompactList}\small\item\em Outer limit of the planet formation area. \item\end{CompactList}\item 
long double {\bf m\-Dust\-Density}\label{class_s_g___stardust_p5}

\begin{CompactList}\small\item\em Stellardust density in the cloud. \item\end{CompactList}\item 
long double {\bf m\-Dust\-Density\-Coeff}\label{class_s_g___stardust_p6}

\begin{CompactList}\small\item\em Density multiplcator factor. \item\end{CompactList}\item 
long double {\bf m\-Cloud\-Eccentricity}\label{class_s_g___stardust_p7}

\begin{CompactList}\small\item\em Stardust cloud eccentricity. \item\end{CompactList}\item 
dust\_\-band $\ast$ {\bf m\-Dust\_\-head}\label{class_s_g___stardust_p8}

\begin{CompactList}\small\item\em List of dustlanes. \item\end{CompactList}\item 
{\bf SG\_\-Star} $\ast$ {\bf m\-Sun}\label{class_s_g___stardust_p9}

\begin{CompactList}\small\item\em The main star of this system. \item\end{CompactList}\item 
int {\bf m\-Max\-Planet}\label{class_s_g___stardust_p10}

\begin{CompactList}\small\item\em The max number of planet in this system. \item\end{CompactList}\item 
{\bf SG\_\-Planet} $\ast$$\ast$ {\bf m\-Planet\-List}\label{class_s_g___stardust_p11}

\begin{CompactList}\small\item\em An Array of planet pointers. \item\end{CompactList}\item 
int {\bf m\-Planet\-Index}\label{class_s_g___stardust_p12}

\begin{CompactList}\small\item\em Index for the planet array. \item\end{CompactList}\end{CompactItemize}


\subsection{Detailed Description}
The cloud of dusts and particles, which is around the star. (Accrete). 

The dust and particles which are surrounding the star at its origine, will accrete to form planets (and moons). This cloud of dust is first composed by one Dust\-Lane, which is quicky sliced by the planets into several dustlanes.

The schematic shows a star with its stellar-dust cloud. Inside the cloud, there is an area where the planets can appear ({\em inner\/} to {\em outer planet bound\/}). A proto-planet is also represented, with its effect area, ie the area where dust will accrete (by gravitation law) to the proto-planet and make it grow.  {\bf Bibliography:} Dole, Stephen H. \char`\"{}Formation of Planetary Systems by Aggregation: a Computer Simulation\char`\"{} October 1969, Rand Corporation Paper P-4226. 



\subsection{Constructor \& Destructor Documentation}
\index{SG_Stardust@{SG\_\-Stardust}!SG_Stardust@{SG\_\-Stardust}}
\index{SG_Stardust@{SG\_\-Stardust}!SG_Stardust@{SG\_\-Stardust}}
\subsubsection{\setlength{\rightskip}{0pt plus 5cm}SG\_\-Stardust::SG\_\-Stardust ({\bf SG\_\-Star} $\ast$ {\em star}, int {\em max\_\-planet} = {\tt 10})}\label{class_s_g___stardust_a0}


Constructor. 

\begin{Desc}
\item[Parameters:]
\begin{description}
\item[{\em star}]A reference to the star of the system. \item[{\em max\_\-planet}]the max number of planet authorized for this system (default=10). /$\ast$ ------------------------------------------------------------------------- \end{description}
\end{Desc}


\subsection{Member Function Documentation}
\index{SG_Stardust@{SG\_\-Stardust}!accreteDust@{accreteDust}}
\index{accreteDust@{accreteDust}!SG_Stardust@{SG\_\-Stardust}}
\subsubsection{\setlength{\rightskip}{0pt plus 5cm}void SG\_\-Stardust::accrete\-Dust (proto\-Planet $\ast$ {\em protoplanet})\hspace{0.3cm}{\tt  [protected]}}\label{class_s_g___stardust_b4}


This function accrete the dust located on the protoplanet orbital path. 

\begin{Desc}
\item[Parameters:]
\begin{description}
\item[{\em protoplanet}]The protoplanet. \end{description}
\end{Desc}
\index{SG_Stardust@{SG\_\-Stardust}!collectDust@{collectDust}}
\index{collectDust@{collectDust}!SG_Stardust@{SG\_\-Stardust}}
\subsubsection{\setlength{\rightskip}{0pt plus 5cm}void SG\_\-Stardust::collect\-Dust (proto\-Planet $\ast$ {\em protoplanet}, dust\_\-band $\ast$ {\em dustband})\hspace{0.3cm}{\tt  [protected]}}\label{class_s_g___stardust_b7}


This function collects Dust from all the existing dustbands, starting at the given dustband. 

\begin{Desc}
\item[Parameters:]
\begin{description}
\item[{\em protoplanet}]The protoplanet that passes thru the dustbands, and grows. \item[{\em dustband}]The first dustband of the list. \end{description}
\end{Desc}
\index{SG_Stardust@{SG\_\-Stardust}!getCriticalMass@{getCriticalMass}}
\index{getCriticalMass@{getCriticalMass}!SG_Stardust@{SG\_\-Stardust}}
\subsubsection{\setlength{\rightskip}{0pt plus 5cm}long double SG\_\-Stardust::get\-Critical\-Mass (proto\-Planet $\ast$ {\em protoplanet})\hspace{0.3cm}{\tt  [protected]}}\label{class_s_g___stardust_b3}


This function returns the mass at which a planet at this orbit will begin to accrete gas and dust. 

\begin{Desc}
\item[Parameters:]
\begin{description}
\item[{\em protoplanet}]The protoplanet (orbit and eccentricity) \end{description}
\end{Desc}
\begin{Desc}
\item[Returns:]The critical mass of the protoplanet (unit = solar masses). \end{Desc}
\index{SG_Stardust@{SG\_\-Stardust}!getInnerEffectLimit@{getInnerEffectLimit}}
\index{getInnerEffectLimit@{getInnerEffectLimit}!SG_Stardust@{SG\_\-Stardust}}
\subsubsection{\setlength{\rightskip}{0pt plus 5cm}long double SG\_\-Stardust::get\-Inner\-Effect\-Limit (proto\-Planet $\ast$ {\em protoplanet})\hspace{0.3cm}{\tt  [protected]}}\label{class_s_g___stardust_b1}


This function returns the inner orbit affected by this proto planet. 

\begin{Desc}
\item[Parameters:]
\begin{description}
\item[{\em protoplanet}]The protoplanet. \end{description}
\end{Desc}
\begin{Desc}
\item[Returns:]The inner effect orbit for this protoplanet (unit = AU). \end{Desc}
\index{SG_Stardust@{SG\_\-Stardust}!getOuterEffectLimit@{getOuterEffectLimit}}
\index{getOuterEffectLimit@{getOuterEffectLimit}!SG_Stardust@{SG\_\-Stardust}}
\subsubsection{\setlength{\rightskip}{0pt plus 5cm}long double SG\_\-Stardust::get\-Outer\-Effect\-Limit (proto\-Planet $\ast$ {\em protoplanet})\hspace{0.3cm}{\tt  [protected]}}\label{class_s_g___stardust_b2}


This function returns the outer orbit affected by this proto planet. 

\begin{Desc}
\item[Parameters:]
\begin{description}
\item[{\em protoplanet}]The protoplanet. \end{description}
\end{Desc}
\begin{Desc}
\item[Returns:]The outer effect orbit for this protoplanet (unit = AU). \end{Desc}
\index{SG_Stardust@{SG\_\-Stardust}!getPlanet@{getPlanet}}
\index{getPlanet@{getPlanet}!SG_Stardust@{SG\_\-Stardust}}
\subsubsection{\setlength{\rightskip}{0pt plus 5cm}{\bf SG\_\-Planet} $\ast$ SG\_\-Stardust::get\-Planet (int {\em index})}\label{class_s_g___stardust_a5}


This function returns a planet (sorted by orbit radius). 

\begin{Desc}
\item[Parameters:]
\begin{description}
\item[{\em index}]The index of the planet in the Planet Array (0..N). \end{description}
\end{Desc}
\begin{Desc}
\item[Returns:]The planet, or NULL if there is no planet with this index. \end{Desc}
\index{SG_Stardust@{SG\_\-Stardust}!isDustAvailable@{isDustAvailable}}
\index{isDustAvailable@{isDustAvailable}!SG_Stardust@{SG\_\-Stardust}}
\subsubsection{\setlength{\rightskip}{0pt plus 5cm}bool SG\_\-Stardust::is\-Dust\-Available (long double {\em inside\_\-range}, long double {\em outside\_\-range})\hspace{0.3cm}{\tt  [protected]}}\label{class_s_g___stardust_b0}


This function indicates if there is still some stellar dust in the specified orbital range. 

\begin{Desc}
\item[Parameters:]
\begin{description}
\item[{\em inside\_\-range}]The inner orbit of the area (unit=UA) \item[{\em outside\_\-range}]The outer orbit of the area (unit=UA) \end{description}
\end{Desc}
\begin{Desc}
\item[Returns:]TRUE if there is still some dust in this area. \end{Desc}
\index{SG_Stardust@{SG\_\-Stardust}!setDustDensityRatio@{setDustDensityRatio}}
\index{setDustDensityRatio@{setDustDensityRatio}!SG_Stardust@{SG\_\-Stardust}}
\subsubsection{\setlength{\rightskip}{0pt plus 5cm}void SG\_\-Stardust::set\-Dust\-Density\-Ratio (long double {\em ratio} = {\tt 1})}\label{class_s_g___stardust_a3}


This function changes the coefficient of density of the dust inthe cloud. 

\begin{Desc}
\item[Parameters:]
\begin{description}
\item[{\em ratio}]The coefficient of density of dust is multiplied by this ratio. \end{description}
\end{Desc}
\index{SG_Stardust@{SG\_\-Stardust}!updateDustLanes@{updateDustLanes}}
\index{updateDustLanes@{updateDustLanes}!SG_Stardust@{SG\_\-Stardust}}
\subsubsection{\setlength{\rightskip}{0pt plus 5cm}void SG\_\-Stardust::update\-Dust\-Lanes (proto\-Planet $\ast$ {\em protoplanet})\hspace{0.3cm}{\tt  [protected]}}\label{class_s_g___stardust_b5}


This function updates the dustbands surrounding the star. 

When a protoplanet passes thru the stellar dust, it creates a new band and takes dust and gas from it. \begin{Desc}
\item[Parameters:]
\begin{description}
\item[{\em protoplanet}]The protoplanet. The protoplanet has a band limited by its inner\-Effect and outer\-Effect, called the \char`\"{}protoplanet'band\char`\"{}. \end{description}
\end{Desc}


The documentation for this class was generated from the following files:\begin{CompactItemize}
\item 
E:/sphinx/LFE/lib\_\-stargen/SG\_\-Stardust.h\item 
E:/sphinx/LFE/lib\_\-stargen/SG\_\-Stardust.cpp\end{CompactItemize}
